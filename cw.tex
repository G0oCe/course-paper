\documentclass[a4paper]{article}
\usepackage[T2A]{fontenc}
\usepackage[utf8]{inputenc}
\usepackage[russian]{babel}
\usepackage[a4paper, margin=2cm]{geometry}
\usepackage{fancyhdr}
\usepackage{hyperref}

\begin{document}

% ------------------ ТИТУЛЬНЫЙ ЛИСТ ------------------
\begin{titlepage}
    \centering
    {\small
    ФИЛИАЛ МГУ имени М. В. ЛОМОНОСОВА в городе БАКУ\\
    ФАКУЛЬТЕТ ПРИКЛАДНОЙ МАТЕМАТИКИ \\} 
    \vfill

    {\Large\textbf{Курсовая работа}}\\[1em]
    {\large\textbf{на тему:}}\\
    \textbf{Шифрование и защита информации в блокчейн-системах в условиях квантовых вычислений}

    \vfill

    \begin{flushright}
    студента III курса группы № 222\\
    Садыгов Алим Хамой оглу\\[1em]
    Научный руководитель:\\
    м.н.с. Сиротич Богдан Михайлович
\end{flushright}

    \vfill
\end{titlepage}

% ------------------ ОГЛАВЛЕНИЕ ------------------
\tableofcontents
\newpage

% ------------------ ОСНОВНАЯ ЧАСТЬ ------------------

\section{Введение}
В работе рассматриваются вопросы шифрования и защиты информации в блокчейн-системах. 
Криптографические методы, такие как \textbf{RSA}, становятся уязвимыми с развитием квантовых вычислений, 
что требует перехода к \underline{постквантовым} алгоритмам.

\subsection{Определения}
\begin{description}
    \item[Блокчейн] последовательная цепочка блоков, содержащих данные и связанных криптографически.
    \item[Блок в блокчейне] структурированная запись в блокчейне, содержащая заголовок (включая хеш предыдущего блока), транзакции и служебную информацию.
    \item[Цифровая подпись] криптографический механизм, подтверждающий подлинность данных.
    \item[Шифрование] преобразование данных с целью их защиты от несанкционированного доступа.
    \item[Асимметричное шифрование] метод, использующий два ключа: открытый для шифрования и закрытый для расшифровывания.
    \item[Открытый ключ] криптографический ключ, предназначенный для шифрования данных или проверки цифровой подписи. Может быть доступен всем пользователям системы.  
    \item[Закрытый ключ] криптографический ключ, используемый для расшифровывания данных или создания цифровой подписи.
    \item[RSA] криптографический алгоритм, основанный на факторизации больших чисел.
    \item[ECDSA (Elliptic Curve Digital Signature Algorithm)] криптографический алгоритм цифровой подписи, использующий свойства эллиптических кривых.
    \item[Алгоритм Шора] алгоритм разложения числа на множители.
    \item[SNDL (Store Now, Decrypt Later)] стратегия сбора данных для их последующей расшифровки.
    \item[Криптографическая стойкость] устойчивость алгоритма к расшифровке без ключа.
    \item[Хеш-функция] криптографический алгоритм, необратимо преобразующий входные данные в фиксированную строку символов.
    \item[Хеш] результат хеш-функции, имеющий фиксированную длину и изменяющийся при изменении входных данных.
\end{description}

\section{Актуальность задачи}

\noindent\textbf{SNDL — основная угроза.}  
Блокчейн-системы используют криптографические алгоритмы, такие как \textbf{RSA} и \textbf{ECDSA}, для защиты данных и цифровой подписи транзакций. Однако стратегия \textbf{SNDL} делает их уязвимыми.

\noindent\textbf{Уязвимость RSA, ECDSA. Алгоритм Шора.}  

\textbf{RSA} основан на сложности факторизации больших чисел, а \textbf{ECDSA} — на вычислительной сложности дискретного логарифмирования на эллиптических кривых. 
\textbf{Алгоритм Шора} ускоряет разложение числа на множители и решает задачу дискретного логарифма на квантовом компьютере, что делает \textbf{RSA} и \textbf{ECDSA} уязвимыми в условиях квантовых атак.

Основные последствия:
\begin{itemize}
    \item \textbf{Кража приватных ключей} — квантовый компьютер сможет восстановить закрытый ключ, что приведёт к подделке подписей транзакций.
    \item \textbf{Компрометация данных} — зашифрованные в прошлом транзакции и контракты будут вскрыты в будущем.
    \item \textbf{Нарушение целостности блокчейна} — возможность подделки подписей ставит под угрозу неизменность записей.
\end{itemize}

\noindent\textbf{Вывод.}  
\begin{enumerate}
    \item SNDL-атаки делают текущие зашифрованные данные уязвимыми в будущем.
    \item RSA и ECDSA становятся незащищёнными в условиях квантовых вычислений.
\end{enumerate}

Таким образом, вопрос безопасности блокчейна в условиях квантовых вычислений остается нерешенным, что требует разработки и внедрения устойчивых криптографических механизмов.
% ------------------ СПИСОК ЛИТЕРАТУРЫ ------------------
\newpage
\begin{center}
\section*{Список литературы}
\end{center}

\vspace{1em}
% сейчас пусто.

\end{document}
