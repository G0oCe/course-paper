\documentclass[a4paper]{article}
\usepackage[T2A]{fontenc}
\usepackage[utf8]{inputenc}
\usepackage[russian]{babel}
\usepackage[a4paper, margin=2cm]{geometry}

\begin{document}

\section{Введение}
В работе рассматриваются вопросы шифрования и защиты информации в блокчейн-системах. 
Криптографические методы, такие как \textbf{RSA}, становятся уязвимыми с развитием квантовых вычислений, 
что требует перехода к \underline{постквантовым} алгоритмам.

\subsection{Определения}
\begin{description}
    \item[Блокчейн] последовательная цепочка блоков, содержащих данные и связанных криптографически.
    \item[Блок в блокчейне] структурированная запись в блокчейне, содержащая заголовок (включая хеш предыдущего блока), транзакции и служебную информацию.
    \item[Цифровая подпись] криптографический механизм, подтверждающий подлинность данных.
    \item[Шифрование] преобразование данных с целью их защиты от несанкционированного доступа.
    \item[Асимметричное шифрование] метод, использующий два ключа: открытый для шифрования и закрытый для расшифровывания. % расшифровывания/расшифровки/расшифрования? что писать? 
    \item[Открытый ключ] криптографический ключ, предназначенный для шифрования данных или проверки цифровой подписи. Может быть доступен всем пользователям системы.  
    \item[Закрытый ключ] криптографический ключ, используемый для расшифровывания данных или создания цифровой подписи.
    \item[RSA] криптографический алгоритм, основанный на факторизации больших чисел.
    \item[Алгоритм Шора] алгоритм разложения числа на множители.
    \item[SNDL (Store Now, Decrypt Later)] стратегия сбора данных для их последующей расшифровки.
    \item[Криптографическая стойкость] устойчивость алгоритма к расшифровке без ключа.
    \item[Хеш-функция] криптографический алгоритм, необратимо преобразующий входные данные в фиксированную строку символов.
    \item[Хеш] результат хеш-функции, имеющий фиксированную длину и изменяющийся при изменении входных данных. %звучит странно, мне не нравится, но не знаю как поменять
    % пара ключей, ключ стоит ли добавлять определение?
\end{description}

\end{document}
