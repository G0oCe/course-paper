\documentclass[a4paper]{article}
\usepackage[T2A]{fontenc}
\usepackage[utf8]{inputenc}
\usepackage[russian]{babel}
\usepackage[a4paper, margin=2cm]{geometry}

\begin{document}

\section{Введение}
В работе рассматриваются вопросы шифрования и защиты информации в блокчейн-системах. 
Криптографические методы, такие как \textbf{RSA}, становятся уязвимыми с развитием квантовых вычислений, 
что требует перехода к \underline{постквантовым} алгоритмам.

\subsection{Определения}
\begin{description}
    \item[Блокчейн] последовательная цепочка блоков, содержащих данные и связанных криптографически.
    \item[Блок в блокчейне] структурированная запись в блокчейне, содержащая заголовок (включая хеш предыдущего блока), транзакции и служебную информацию.
    \item[Цифровая подпись] криптографический механизм, подтверждающий подлинность данных.
    \item[Шифрование] преобразование данных с целью их защиты от несанкционированного доступа.
    \item[Асимметричное шифрование] метод, использующий два ключа: открытый для шифрования и закрытый для расшифровывания. % расшифровывания/расшифровки/расшифрования? что писать? 
    \item[Открытый ключ] криптографический ключ, предназначенный для шифрования данных или проверки цифровой подписи. Может быть доступен всем пользователям системы.  
    \item[Закрытый ключ] криптографический ключ, используемый для расшифровывания данных или создания цифровой подписи.
    \item[RSA] криптографический алгоритм, основанный на факторизации больших чисел.
    \item[ECDSA (Elliptic Curve Digital Signature Algorithm)] криптографический алгоритм цифровой подписи, использующий свойства эллиптических кривых.
    \item[Алгоритм Шора] алгоритм разложения числа на множители.
    \item[SNDL (Store Now, Decrypt Later)] стратегия сбора данных для их последующей расшифровки.
    \item[Криптографическая стойкость] устойчивость алгоритма к расшифровке без ключа.
    \item[Хеш-функция] криптографический алгоритм, необратимо преобразующий входные данные в фиксированную строку символов.
    \item[Хеш] результат хеш-функции, имеющий фиксированную длину и изменяющийся при изменении входных данных. %звучит странно, мне не нравится, но не знаю как поменять
    % пара ключей, ключ стоит ли добавлять определение?
\end{description}

\section{Актуальность задачи}

\noindent\textbf{SNDL — основная угроза.}  
Блокчейн-системы используют криптографические алгоритмы, такие как \textbf{RSA} и \textbf{ECDSA}, для защиты данных и цифровой подписи транзакций. Но \textbf{SNDL}, делает их не безопасными.

% решил убрать так как ниже я рассказываю про основные риски
%Основные риски SNDL:  
%\begin{description}

%   \item \textbf{Кража приватных ключей} — при восстановлении закрытого ключа можно подделывать подписи транзакций.
%    \item \textbf{Компрометация данных} — зашифрованные в прошлом транзакции и контракты будут вскрыты в будущем.
%    \item \textbf{Нарушение целостности блокчейна} — возможность подделки подписи ставит под угрозу неизменность записей.
%\end{description}




% Как будто бы лишнее, лично кажется что стоит убрать это
%\noindent\textbf{Переход на постквантовую криптографию (Ethereum).}  
%Блокчейн-платформы осознают угрозу квантовых атак и разрабатывают механизмы защиты. \textbf{Ethereum 2.0} рассматривает замену \textbf{ECDSA} на постквантовые алгоритмы подписи, такие как \textbf{Dilithium} и \textbf{Falcon}. Однако:
%\begin{itemize}
%    \item Переход требует значительных изменений в протоколе.
%    \item Квантово-устойчивые алгоритмы медленнее традиционных методов.
%    \item Отсутствует единый стандарт для квантовой защиты блокчейнов.
%\end{itemize}
%
\noindent\textbf{Уязвимость RSA, ECDSA. Алгоритм Шора.}  

\textbf{RSA} основан на сложности факторизации больших чисел, а \textbf{ECDSA} — на вычислительной сложности дискретного логарифмирования на эллиптических кривых. 
\textbf{Алгоритм Шора} ускоряет разложение числа на множители и решает задачу дискретного логарифма на квантовом компьютере, что делает \textbf{RSA} и \textbf{ECDSA} уязвимыми в условиях квантовых атак.


Основные последствия:
\begin{itemize}
    \item При взломе \textbf{RSA и ECDSA} можно получить приватные ключи пользователей.
    \item \textbf{Кража приватных ключей} — при восстановлении закрытого ключа можно подделывать подписи транзакций.
    \item \textbf{Компрометация данных} — зашифрованные в прошлом транзакции и контракты будут вскрыты в будущем.
    \item \textbf{Нарушение целостности блокчейна} — возможность подделки подписи ставит под угрозу неизменность записей.
\end{itemize}

\noindent\textbf{Вывод.}  
\begin{enumerate}
    \item SNDL-атаки делают текущие зашифрованные данные уязвимыми в будущем.
\end{enumerate}

Таким образом, вопрос безопасности блокчейна в условиях квантовых вычислений остается нерешенным, что требует разработки и внедрения устойчивых криптографических механизмов.


\end{document}

% НЕ ПОНЯТНО
1. О чем вообще моя курсовая(
2. Обязательно ли говорить про квантовые компьютеры? Ведь по сути я просто реализую
алгоритм Шора и покажу как меняется время от длинны ключа RSA
3. Какие выводы я получу, имеют ли они смысл? 
4. Не нужно ли предложить метод решения проблем SNDL и варианты оптимиазии 
алгоритма Шора (смелое заявление) 
